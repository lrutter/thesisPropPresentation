\documentclass{bredelebeamer}

\renewcommand<>{\item}{\beameroriginal\item\vspace{\stretch{.1}}}
\usefonttheme{professionalfonts} % using non standard fonts for beamer
\usefonttheme{serif} % default family is serif
\usepackage[T1]{fontenc}
\usepackage{lmodern}

\setbeamertemplate{itemize/enumerate body begin}{\Large}
\setbeamertemplate{itemize/enumerate subbody begin}{\large}

%%%%%%%%%%%%%%%%%%%%%%%%%%%%%%%%%%%%%%%%%%%%%%%%

\title[Ph.D. Thesis Proposal]{\huge Visualization methods for genealogical and RNA-sequencing datasets}

\author[Lindsay Rutter (ISU)]{\Large Ph.D. Thesis Proposal\\ Lindsay Rutter\\ \vspace{9mm} \normalsize \textbf{Program of Study Committee:}\\ Dianne Cook (Major Professor)\\ Amy Toth (Major Professor)\\ Heike Hofmann\\ Daniel Nettleton\\ James Reecy}


\date[May 16, 2016]{\small May 16, 2016}

\subject{Sujet de votre diaporama}
% C'est utilisé dans les métadonnes du PDF

%%%%%%%%%%%%%%%%%%%%%%%%%%%%%%%%%%%%%%%%%%%%%%%%%%%%%%%%%%%%%%%%%%%%%
\begin{document}

\begin{frame}
  \titlepage
\end{frame}

\begin{frame}{Sommaire}
  \tableofcontents
  % possibilité d'ajouter l'option [pausesections]
\end{frame}

%%%%%%%%%%%%%%%%%%%%%%%%%%%%%%%%%%%%%% Background %%%%%%%%%%%%%%%%%%%%%%%%%%%%%
%\section{My Background}

\begin{frame}{My Background}

\begin{itemize}
		\item Education
		\begin{itemize}
		  \item B.S. in Bioengineering\\ Pennsylvania State University (2003-2007)
		  \item Major in Bioinformatics and Computational Biology\\ Iowa State University (2012-Present)
		  \item Minor in Statistics\\ Iowa State University (2012-Present)
		\end{itemize}
		\item Internships
		\begin{itemize}
		  \item Okinawa Institute of Science and Technology (Summer 2014)
		  \item MathWorks (Summer 2016)
		\end{itemize}
	\end{itemize}
\end{frame}

%%%%%%%%%%%%%%%%%%%%%%%%%%%%%%%%%%%%%% Introduction %%%%%%%%%%%%%%%%%%%%%%%%%%%%%
\section{Introduction}

\begin{frame}{Motivation}
  \begin{itemize}
		\item Use visualization to explore data, check data quality, assess model diagnostics, and compare results across methods
		\item Problem: Limited choice of plots
		\begin{itemize}
		  \item Solution: Develop new plots
		\end{itemize}
		\item Problem: Large datasets
		\begin{itemize}
		  \item Solution: Improve computational expense
		  \item Solution: Repair overplotting issues
		  \item Solution: Enhance pattern detection methods
		  \item Solution: Incorporate interactive graphics
		\end{itemize}
	\end{itemize}
\end{frame}

\begin{frame}{Previous research}
  \begin{itemize}
    \item Current static software: \texttt{pedigree} (Coster 2013), \texttt{kinship2} (Therneau et al. 2015), \texttt{GraphViz} (Gansner and North 2000), \texttt{Cytoscape} (Shannon et al. 2003), \texttt{ggplot2} (Wickham 2009), \texttt{GGally} (Schloerke et al. 2016), \texttt{nullabor} (Wickham et al. 2014),
    \item Current interactive software: \texttt{GGobi} (Swayne et al. 2003), \texttt{tourr} (Wickham et al. 2011)
    
    \item Seminal work: Parallel coordinate plots (Inselberg 1985, Wegman 1990), Replicate line plots (Cook et al. 2007)
    \item Seminal work2 (use graphical and numerical): (Lawrence et al. 2008, Yin et al. 2013, Yin et al. 2012), Clustering analysis has been applied to plant breeding purposes in previous studies (Newell et al. 2013), RNA-sequencing (DESeq2, edgeR), \texttt{limma} (Ritchie et al. 2015), \texttt{edgeR} (Robinson et al. 2010), \texttt{RUVseq} (Risso et al. 2014), \texttt{ggbio} (Yin et al. 2012), use nullabor for RNAseq (Yin et al. 2013)
	\end{itemize}
\end{frame}







%%%%%%%%%%%%%%%%%%%%%%%%%%%%%%%%%%%%%% section 2 %%%%%%%%%%%%%%%%%%%%%%%%%%%%%
\section{Blocks}

\begin{frame}{Blocks}

\begin{block}{Bloc simple}
\begin{itemize}
\item Premier point
\item Second point
\item Troisième point
\end{itemize}
\end{block}

\begin{exampleblock}{Bloc exemple}
\begin{itemize}
\item Premier point
\item Second point
\item Troisième point
\end{itemize}
\end{exampleblock}

\begin{alertblock}{Bloc alert}
\begin{itemize}
\item Premier point
\item Second point
\item Troisième point
\end{itemize}
\end{alertblock}
\end{frame}

%%%%%%%%%%%%%%%%%%%%%%%%%%%%%%%%%%%%%% section 3 %%%%%%%%%%%%%%%%%%%%%%%%%%%%%
\section{Boxes}

\begin{frame}{Boxes}

\begin{columns}

\begin{column}{0.5\textwidth}
\boitejaune{
Ceci est \\
une boite jaune
}

\boiteorange{
Ceci est \\
une boite orange
}

\boitemarron{
Ceci est \\
une boite marron
}
\end{column}

\begin{column}{0.5\textwidth}
\boiteviolette{
Ceci est \\
une boite violette
}

\boitebleue{
Ceci est \\
une boite bleue
}

\boitegrise{
Ceci est \\
une boite grise
}

\end{column}

\end{columns}


\end{frame}

%%%%%%%%%%%%%%%%%%%%%%%%%%%%%%%%%%%%%% section 4 %%%%%%%%%%%%%%%%%%%%%%%%%%%%%
\section{Text}

\begin{frame}{Titre de la frame} 

Voici du texte normal

\alert{Voici du texte \texttt{alert}}

\exemple{Voici du texte \texttt{exemple}}

\emph{Voici du texte \texttt{emphase}}

\end{frame}

%%%%%%%%%%%%%%%%%%%%%%%%%%%%%%%%%%%%%% section 6 %%%%%%%%%%%%%%%%%%%%%%%%%%%%%
\section{Tables}

\begin{frame}{Tables}

% merci: http://tex.stackexchange.com/questions/112343/beautiful-table-samples

\begin{tcolorbox}[tabjaune,tabularx={X||Y|Y|Y|Y||Y}, boxrule=0.5pt]
Couleur & Prix 1  & Prix 2  & Prix 3   & Prix 4   & Prix 5 \\\hline\hline
Rouge   & 10.00   & 20.00   &  30.00   &  40.00   & 100.00 \\\hline
Vert    & 20.00   & 30.00   &  40.00   &  50.00   & 140.00 \\\hline
Bleu    & 30.00   & 40.00   &  50.00   &  60.00   & 180.00 \\\hline\hline
Orange  & 60.00   & 90.00   & 120.00   & 150.00   & 420.00
\end{tcolorbox}

\begin{tcolorbox}[tabvert,tabularx={X||Y|Y|Y|Y||Y}, boxrule=0.5pt, title=Mon tableau des prix]
Couleur & Prix 1  & Prix 2  & Prix 3   & Prix 4   & Prix 5 \\\hline\hline
Rouge   & 10.00   & 20.00   &  30.00   &  40.00   & 100.00 \\\hline
Vert    & 20.00   & 30.00   &  40.00   &  50.00   & 140.00 \\\hline
Bleu    & 30.00   & 40.00   &  50.00   &  60.00   & 180.00 \\\hline\hline
Orange  & 60.00   & 90.00   & 120.00   & 150.00   & 420.00
\end{tcolorbox}

\end{frame}


\begin{frame}{Tables}

% merci: http://tex.stackexchange.com/questions/112343/beautiful-table-samples

\begin{tcolorbox}[tabgris,tabularx={X||Y|Y|Y|Y||Y}, boxrule=0.5pt]
Couleur & Prix 1  & Prix 2  & Prix 3   & Prix 4   & Prix 5 \\\hline\hline
Rouge   & 10.00   & 20.00   &  30.00   &  40.00   & 100.00 \\\hline
Vert    & 20.00   & 30.00   &  40.00   &  50.00   & 140.00 \\\hline
Bleu    & 30.00   & 40.00   &  50.00   &  60.00   & 180.00 \\\hline\hline
Orange  & 60.00   & 90.00   & 120.00   & 150.00   & 420.00
\end{tcolorbox}

\begin{tcolorbox}[taborange,tabularx={X||Y|Y|Y|Y||Y}, boxrule=0.5pt, title=Mon tableau des prix]
Couleur & Prix 1  & Prix 2  & Prix 3   & Prix 4   & Prix 5 \\\hline\hline
Rouge   & 10.00   & 20.00   &  30.00   &  40.00   & 100.00 \\\hline
Vert    & 20.00   & 30.00   &  40.00   &  50.00   & 140.00 \\\hline
Bleu    & 30.00   & 40.00   &  50.00   &  60.00   & 180.00 \\\hline\hline
Orange  & 60.00   & 90.00   & 120.00   & 150.00   & 420.00
\end{tcolorbox}

\end{frame}

%%%%%%%%%%%%%%%%%%%%%%%%%%%%%%%%%%%%%% section 7 %%%%%%%%%%%%%%%%%%%%%%%%%%%%%
\section{Images}

\begin{frame}{Titre de la frame} 

\begin{figure}
\centering
\includegraphics[scale=0.5]{images/architecturebretonne_wikipedia.jpg}
\caption{Éléments d'architecture bretonne typique du Sud de la France.
(\href{http://commons.wikimedia.org/wiki/File:Colmar_-_Alsace.jpg}{Wikipédia.fr} CC-By-Sa)}
\end{figure}
\end{frame}



\end{document}

