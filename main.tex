\documentclass{bredelebeamer}

\renewcommand<>{\item}{\beameroriginal\item\vspace{\stretch{.1}}}
\usefonttheme{professionalfonts} % using non standard fonts for beamer
\usefonttheme{serif} % default family is serif
\usepackage[T1]{fontenc}
\usepackage{lmodern}

\setbeamertemplate{itemize/enumerate body begin}{\Large}
\setbeamertemplate{itemize/enumerate subbody begin}{\large}
\setbeamertemplate{itemize/enumerate subsubbody begin}{\large}

%%%%%%%%%%%%%%%%%%%%%%%%%%%%%%%%%%%%%%%%%%%%%%%%

\title[Ph.D. Thesis Proposal]{\huge Visualization methods for genealogical and RNA-sequencing datasets}

\author[Lindsay Rutter (ISU)]{\Large Ph.D. Thesis Proposal\\ Lindsay Rutter\\ \vspace{9mm} \normalsize \textbf{Program of Study Committee:}\\ Dianne Cook (Major Professor)\\ Amy Toth (Major Professor)\\ Heike Hofmann\\ Daniel Nettleton\\ James Reecy}


\date[May 16, 2016]{\small May 16, 2016}

\subject{Sujet de votre diaporama}
% C'est utilisé dans les métadonnes du PDF

%%%%%%%%%%%%%%%%%%%%%%%%%%%%%%%%%%%%%%%%%%%%%%%%%%%%%%%%%%%%%%%%%%%%%
\begin{document}

\begin{frame}
  \titlepage
\end{frame}

\begin{frame}{Sommaire}
  \tableofcontents
  % possibilité d'ajouter l'option [pausesections]
\end{frame}

%%%%%%%%%%%%%%%%%%%%%%%%%%%%%%%%%%%% BACKGROUND %%%%%%%%%%%%%%%%%%%%%%%%%%%%%%%%%%%%%
%\section{My Background}

\begin{frame}{My Background}

\begin{itemize}
		\item Education
		\begin{itemize}
		  \item B.S. in Bioengineering\\ \textit{Pennsylvania State University} (2003-2007)
		  \item Major in Bioinformatics and Computational Biology\\ \textit{Iowa State University} (2012-Present)
		  \item Minor in Statistics\\ \textit{Iowa State University} (2012-Present)
		\end{itemize}
		\item Internships
		\begin{itemize}
		  \item Okinawa Institute of Science and Technology (Summer 2014)
		  \item MathWorks (Summer 2016)
		\end{itemize}
	\end{itemize}
\end{frame}

%%%%%%%%%%%%%%%%%%%%%%%%%%%%%%%%%%%% INTRODUCTION %%%%%%%%%%%%%%%%%%%%%%%%%%%%%%%%%%%% 
\section{Introduction}

\begin{frame}{Motivation}
  \begin{itemize}
		\item Use visualization to explore data, check data quality, assess model diagnostics, and compare results across methods
		\item Problem: Limited choice of plots
		\begin{itemize}
		  \item Solution: Develop new plots
		\end{itemize}
		\item Problem: Large datasets
		\begin{itemize}
		  \item Solution: Improve computational expense
		  \item Solution: Repair overplotting issues
		  \item Solution: Enhance pattern detection methods
		  \item Solution: Incorporate interactive graphics
		\end{itemize}
	\end{itemize}
\end{frame}

\begin{frame}{Previous research}
  \begin{itemize}
    \item \textbf{Static visualization software:} \texttt{ggplot2} (Wickham 2009), \texttt{GGally} (Schloerke et al. 2016), \texttt{nullabor} (Wickham et al. 2014), \texttt{ggbio} (Yin et al. 2012)
    \item \textbf{Interactive visualization software:} \texttt{GGobi} (Swayne et al. 2003), \texttt{tourr} (Wickham et al. 2011), \texttt{plotly} (Sievert et al. 2016)
    \item \textbf{General visualization:} Parallel coordinate plots (Inselberg 1985, Wegman 1990), Visual statistical inference (Chowdhury et al. 2015)
	\end{itemize}
\end{frame}

\begin{frame}{Previous research}
  \begin{itemize}
    \item \textbf{Genealogical visualization:} \texttt{pedigree} (Coster 2013), \texttt{kinship2} (Therneau et al. 2015), \texttt{GraphViz} (Gansner and North 2000), \texttt{Cytoscape} (Shannon et al. 2003)
    \item \textbf{Gene expression visualization:} \texttt{explorase} (Lawrence et al. 2008), \texttt{limma} (Ritchie et al. 2015), \texttt{edgeR} (Robinson et al. 2010), \texttt{DESeq2} (Love at al. 2014), \texttt{RUVseq} (Risso et al. 2014)
    \item \textbf{Gene expression visual inference:} (Yin et al. 2013)
    \item \textbf{Biological clustering:} (Newell et al. 2013)
	\end{itemize}
\end{frame}

\begin{frame}{Problems}
  \begin{itemize}
    \item Standard genealogical plots can be ambiguous
    \item Popular RNA-seq visualization tools are misleading
    \item Time and space constraints in large RNA-seq data
	\end{itemize}
\end{frame}

\begin{frame}{Thesis proposal overview}
  \begin{itemize}
    \item (Chapter 2) Visualizing genealogical data
      \begin{itemize}
        \item Goal: Create unambiguous genealogy visualization plots, adapt genealogical plots for large datasets, incorporate interactive genealogical plots 
      \end{itemize}
    \item (Chapter 3) Visualizing clustering analysis of RNA-seq data
      \begin{itemize}
        \item Goal: Develop tools to visualize and interact with gene clusterings to determine genes of interest
      \end{itemize}
    \item (Chapter 4) Visualizing significance tests of RNA-seq data
      \begin{itemize}
        \item Goal: Develop tools to visualize, interact with, and permute differentially expressed genes from significance testing
      \end{itemize}
	\end{itemize}
\end{frame}

%%%%%%%%%%%%%%%%%%%%%%%%%%%%%%%%% GENEALOGY %%%%%%%%%%%%%%%%%%%%%%%%%%%%%
\section{Visualizing genealogy}

\begin{frame}
\begin{center}
\fcolorbox{black}{titleColor}{
\begin{minipage}{\textwidth}
\begin{center}
\huge Visualization methods for genealogical \\
datasets
\end{center}
\end{minipage}
}
\end{center}
\end{frame}

\begin{frame}{Genealogy}
\begin{itemize}
  \item Study of parent-child relationships
  \item Provides tools to better understand traits that arise in lineages
  \begin{itemize}
    \item Desirable (disease resistance)
    \item Undesirable (hemophilia)
  \end{itemize}
  \item Can be represented visually
\end{itemize}
\end{frame}

\begin{frame}{Current visual tools}
\begin{columns}
\begin{column}{0.6\textwidth}
\begin{figure}
\centering
\fbox{\includegraphics[scale=0.38]{kinshipFig.png}}
\caption{\texttt{kinship2}: Ambiguous position of green node, who is both second and third generation}
\end{figure}
\end{column}
\begin{column}{0.5\textwidth}
\begin{figure}
\centering
\fbox{\includegraphics[scale=0.28]{Graph.png}}
\caption{\texttt{Cytoscape} and \texttt{GraphViz}: Ambiguous position of green node, who is both second and third generation}
\end{figure}
\end{column}
\end{columns}
\end{frame}

\begin{frame}{ggenealogy}
\begin{itemize}
  \item \textbf{ggenealogy:}\texttt{R} package to visualize genealogical structures
  \item First example data is soybean genealogy
  \begin{itemize}
    \item Soybean variety data collected from
      \begin{itemize}
        \item Field trials
        \item Genetic studies
        \item USDA bulletins
      \end{itemize}
    \item Data frame of 412 rows (parent-child relations)
    \item Each variety (n=230)
      \begin{itemize}
        \item Developmental years
        \item Copy number variants (CNV)
        \item Single nucleotide polymorphisms (SNPs)
        \item Protein content and yield
      \end{itemize}
  \end{itemize}
\end{itemize}
\end{frame}

\begin{frame}  
\begin{figure}
\centering
\fbox{\includegraphics[scale=0.55]{LeeAD3.png}}
\caption{\texttt{ggenealogy}: Solution to ambiguous positions of nodes}
\end{figure}
\end{frame}

\begin{frame}{Plot shortest path}
\begin{figure}
\centering
\fbox{\includegraphics[scale=0.45]{pathTNZB.png}}
\caption{\textbf{Left:} The shortest path between Tokyo and Narow is composed of a sequence of parent-child relationships. \textbf{Right:} The shortest path between Zane and Bedford instead have a cousin-like relationship.}
\end{figure}
\end{frame}

\begin{frame}{Superimpose path on full structure}
\begin{figure}
\centering
\fbox{\includegraphics[scale=0.45]{plotTNBin3.png}}
\caption{The shortest path between Tokyo and Narow, superimposed over the data structure, using a bin size of 3.}
\end{figure}
\end{frame}

\begin{frame}{Superimpose path on full structure}
\begin{figure}
\centering
\fbox{\includegraphics[scale=0.45]{plotTNBin6.png}}
\caption{The shortest path between Tokyo and Narow, superimposed over the data structure, using a bin size of 6.}
\end{figure}
\end{frame}

\begin{frame}{ggenealogy}
\begin{itemize}
  \item Second example data is genealogy of academic statisticians
  \begin{itemize}
    \item Math Genealogy Project
      \begin{itemize}
        \item Web database of genealogy of academic mathematicians
        \item North Dakota State University Department of Mathematics and the American Mathematical Society
        \item Queried for people with advanced degree in "Statistics" with parent with advanced degree in "Statistics"
      \end{itemize}
    \item Data frame of 8165 rows (3291 parent-child relations)
    \item Each individual (n=7122)
      \begin{itemize}
        \item Year of degree acquisition
        \item Country of degree acquisition
        \item School of degree acquisition
        \item Thesis title
      \end{itemize}
  \end{itemize}
\end{itemize}
\end{frame}

\begin{frame}[fragile]
\frametitle{Including Code}
\begin{columns}
\begin{column}{0.5\textwidth}
\begin{semiverbatim}
> statUI <- statGeneal[which(statGeneal
$school=="University of Iowa"),]$child
> length(statUI)
[1] 54
> numDUI <- sapply(statUI, dFunc)
> table(numDUI)
numDUI
 0  1  7 25 
48  4  1  1 
> which(numDUI==25)
Edward Wegman
> which(numDUI==7)
Daniel Nettleton
\end{semiverbatim}
\end{column}
\begin{column}{0.5\textwidth}
\begin{figure}
\centering
\fbox{\includegraphics[scale=0.5]{dNett.png}}
\caption{Seven "descendents" of Dr. Nettleton.}
\end{figure}
\end{column}
\end{columns}
\end{frame}




\begin{frame}{Blocks}

\begin{block}{Bloc simple}
\begin{itemize}
\item Premier point
\item Second point
\item Troisième point
\end{itemize}
\end{block}

\begin{exampleblock}{Bloc exemple}
\begin{itemize}
\item Premier point
\item Second point
\item Troisième point
\end{itemize}
\end{exampleblock}

\begin{alertblock}{Bloc alert}
\begin{itemize}
\item Premier point
\item Second point
\item Troisième point
\end{itemize}
\end{alertblock}
\end{frame}

%%%%%%%%%%%%%%%%%%%%%%%%%%%%%%%%%%%%%% section 3 %%%%%%%%%%%%%%%%%%%%%%%%%%%%%
\section{Boxes}

\begin{frame}{Boxes}

\begin{columns}

\begin{column}{0.5\textwidth}
\boitejaune{
Ceci est \\
une boite jaune
}

\boiteorange{
Ceci est \\
une boite orange
}

\boitemarron{
Ceci est \\
une boite marron
}
\end{column}

\begin{column}{0.5\textwidth}
\boiteviolette{
Ceci est \\
une boite violette
}

\boitebleue{
Ceci est \\
une boite bleue
}

\boitegrise{
Ceci est \\
une boite grise
}

\end{column}

\end{columns}


\end{frame}

%%%%%%%%%%%%%%%%%%%%%%%%%%%%%%%%%%%%%% section 4 %%%%%%%%%%%%%%%%%%%%%%%%%%%%%
\section{Text}

\begin{frame}{Titre de la frame} 

Voici du texte normal

\alert{Voici du texte \texttt{alert}}

\exemple{Voici du texte \texttt{exemple}}

\emph{Voici du texte \texttt{emphase}}

\end{frame}

%%%%%%%%%%%%%%%%%%%%%%%%%%%%%%%%%%%%%% section 6 %%%%%%%%%%%%%%%%%%%%%%%%%%%%%
\section{Tables}

\begin{frame}{Tables}

% merci: http://tex.stackexchange.com/questions/112343/beautiful-table-samples

\begin{tcolorbox}[tabjaune,tabularx={X||Y|Y|Y|Y||Y}, boxrule=0.5pt]
Couleur & Prix 1  & Prix 2  & Prix 3   & Prix 4   & Prix 5 \\\hline\hline
Rouge   & 10.00   & 20.00   &  30.00   &  40.00   & 100.00 \\\hline
Vert    & 20.00   & 30.00   &  40.00   &  50.00   & 140.00 \\\hline
Bleu    & 30.00   & 40.00   &  50.00   &  60.00   & 180.00 \\\hline\hline
Orange  & 60.00   & 90.00   & 120.00   & 150.00   & 420.00
\end{tcolorbox}

\begin{tcolorbox}[tabvert,tabularx={X||Y|Y|Y|Y||Y}, boxrule=0.5pt, title=Mon tableau des prix]
Couleur & Prix 1  & Prix 2  & Prix 3   & Prix 4   & Prix 5 \\\hline\hline
Rouge   & 10.00   & 20.00   &  30.00   &  40.00   & 100.00 \\\hline
Vert    & 20.00   & 30.00   &  40.00   &  50.00   & 140.00 \\\hline
Bleu    & 30.00   & 40.00   &  50.00   &  60.00   & 180.00 \\\hline\hline
Orange  & 60.00   & 90.00   & 120.00   & 150.00   & 420.00
\end{tcolorbox}

\end{frame}


\begin{frame}{Tables}

% merci: http://tex.stackexchange.com/questions/112343/beautiful-table-samples

\begin{tcolorbox}[tabgris,tabularx={X||Y|Y|Y|Y||Y}, boxrule=0.5pt]
Couleur & Prix 1  & Prix 2  & Prix 3   & Prix 4   & Prix 5 \\\hline\hline
Rouge   & 10.00   & 20.00   &  30.00   &  40.00   & 100.00 \\\hline
Vert    & 20.00   & 30.00   &  40.00   &  50.00   & 140.00 \\\hline
Bleu    & 30.00   & 40.00   &  50.00   &  60.00   & 180.00 \\\hline\hline
Orange  & 60.00   & 90.00   & 120.00   & 150.00   & 420.00
\end{tcolorbox}

\begin{tcolorbox}[taborange,tabularx={X||Y|Y|Y|Y||Y}, boxrule=0.5pt, title=Mon tableau des prix]
Couleur & Prix 1  & Prix 2  & Prix 3   & Prix 4   & Prix 5 \\\hline\hline
Rouge   & 10.00   & 20.00   &  30.00   &  40.00   & 100.00 \\\hline
Vert    & 20.00   & 30.00   &  40.00   &  50.00   & 140.00 \\\hline
Bleu    & 30.00   & 40.00   &  50.00   &  60.00   & 180.00 \\\hline\hline
Orange  & 60.00   & 90.00   & 120.00   & 150.00   & 420.00
\end{tcolorbox}

\end{frame}

%%%%%%%%%%%%%%%%%%%%%%%%%%%%%%%%%%%%%% section 7 %%%%%%%%%%%%%%%%%%%%%%%%%%%%%
\section{Images}

\begin{frame}{Titre de la frame} 

\begin{figure}
\centering
\includegraphics[scale=0.5]{images/architecturebretonne_wikipedia.jpg}
\caption{Éléments d'architecture bretonne typique du Sud de la France.
(\href{http://commons.wikimedia.org/wiki/File:Colmar_-_Alsace.jpg}{Wikipédia.fr} CC-By-Sa)}
\end{figure}
\end{frame}



\end{document}

